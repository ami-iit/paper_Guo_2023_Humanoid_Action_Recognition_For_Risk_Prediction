
\section{CONCLUSIONS}
\label{sec:conclusions}
% conclusions of proposed system
In this paper, we presented a framework that integrates wearable sensing, human state estimation, human action/motion prediction and NIOSH index for real-time manual lifting applications. Through online recognition of human actions, the execution of a single lifting activity can be segmented into a series of continuous parts. The commencement of each sub-action is considered the initial human state, with subsequent moments within this sub-action being regarded as temporary destination states. With the help of motion prediction, future human status can also be obtained. Hence RNLE can be applied to assess risks within the predicted time horizon. The vibrotactile feedback enables anticipated alert on the predicted lifting risks. The performance of the framework is tested in an experimental lifting scenario using the iFeel wearable system.

% future work concerning the limitations
Future work should first address the problem of generalization by expanding the current lifting data set, such that more complex realistic lifting tasks can be considered. By improving the performance of GMoE model, a more precise retrieval of NIOSH geometry variables could be expected. It would also be interesting to include upper trunk twisting and overhead lifting in order to utilize the NIOSH equation. Moreover, a learning-based ergonomics assessment approach could be another promising topic.  